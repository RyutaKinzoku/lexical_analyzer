\documentclass [xcolor=svgnames, t] {beamer} 
\usepackage[utf8]{inputenc}
\usepackage{booktabs, comment} 
\usepackage[absolute, overlay]{textpos} 
\usepackage{pgfpages}
\usepackage[font=footnotesize]{caption}

\setbeamercolor{author in head/foot}{bg=darkcyan}
\setbeamertemplate{page number in head/foot}{}
\usepackage{csquotes}


\usepackage{amsmath}
\usepackage[makeroom]{cancel}


\usepackage{textpos}

\usepackage{tikz}

\usetheme{Madrid}
\definecolor{darkcyan}{RGB}{0, 139, 139}
\usecolortheme[named=darkcyan]{structure}
\usepackage{tikz}



\title[Lexical Analysis]{Lexical Analysis}
\subtitle{C Language}
\institute[]{Department of Earth, Environmental, and Planetary Sciences  \\Brown University}
\author[Project 1]{
	Reyner Marxell Arias Muñoz,
	Kenneth Ibarra Vargas,
	David Benavides Naranjo}
\institute[]{Project 1, Compilers and Interpreters course, I 2022 Semester}
\date{\today}

\begin{document}
\begin{frame}
\maketitle
\end{frame}

\section{Introduction}
\begin{frame}{Scanning}
   Como el archivo fuente tiene directivas include y define, el preprocesador previamente aplico correctamente cada una de ellas para que asi el escaner reciba solamente un archivo temporal de entrada.
   \\Despues, el escaner recorre el archivo temporal devolviendo los tokens en orden y uno por uno, segun el programa fuente analizado. En c existen token de diferentes tipos como operadores, identificadores, literales, palabras reservadas y caracteres separadores.
   \\Cualquier caracter no perteneciente al lexico de c que pueda llegar a ser analizado es retornado como error lexico. 
\end{frame}
\section{Literature Review}

\section{Methodology}

\section{Experiment}

\section{Conclusion}
    

\begin{frame} [allowframebreaks]\frametitle{References} 
    \bibliographystyle{apalike}
    \bibliography{bibfile}
\end{frame}

\end{document}